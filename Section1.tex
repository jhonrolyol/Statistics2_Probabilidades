% Section 1.- Introducción
\section{Introducción}
    La distribución de frecuencias de las observaciones de un fenómeno 
    casual es un recurso muy poderoso para entender la variación del mismo.
    Sin embargo, haciendo suposiciones apropiadas sobre el fenómeno en esudio
    y sin observar directamente el mismo, podemos construir un modelo teórico 
    que represente en forma adecuada a la distribución de frecuencias cuando el fenómeno 
    es observado directamente. Tales modelos son llamados modelos de probabilidad.

    Por consiguiente, en estas notas de lectura se desarrollan las nociones básicas 
    y propiedades del modelo propuesto que serán utilizadas 
    en la inferencia estadística.
    % Subsection 1.1 .- Experimento aleatorio
    \subsection{Experimento Aleatorio}
    Con la finalidad de facilitar la comprensión de lo que sigue, definiremos algunos términos.
        % Definición 1.1
        \begin{defi}
            Un experimento es un proceso mediante el cual se obtiene un resultado 
            de una observación. Un experimento puede ser determinístico y no determinístico.
        \end{defi}
        % Definición 1.2
        \begin{defi}
            Un experimento es determinístico cuando el resultado de la
            observación es determinado en forma precisa por las condiciones 
            bajo las cuales se realiza dicho experimento.
        \end{defi}
        % Definición 1.3
        \begin{defi}
            Un experimento es aleatorio o no determinístico cuando los resultados 
            de la observación no se pueden predecir con exactitud antes de realizar
            el experimento.
        \end{defi}
    A continuación, se muestra algunos ejemplos de experimentos determinísticos.
        \begin{itemize}
            \item Observar la suma de dos números naturales pares.
            \item Observar el color de una bola extraída de una urna que contiene solamente bolas negras.
            \item Observar el área barrida durante cierto periodo de tiempo por planetas obedeciendo \textcolor{blue}{leyes de Kepler} \footnote{Las leyes de Kepler fueron enunciadas por Johannes Kepler para describir matemáticamente el movimiento de los planetas en sus órbitas alrededor del Sol. Fuente: \textcolor{red!75}{\url{https://es.wikipedia.org/wiki/Leyes_de_Kepler}}}.
        \end{itemize}
    Asimismo, se tienen ejemplos de experimentos no determinísticos o aleatorios.
        \begin{itemize}
            \item Lanzar un dado y observar el número que aparece en la cara superior.
            \item Lanzar una moneda 8 veces y observar las sucesión de caras y sellos.
            \item Lanzar una moneda 6 veces y observar el número de caras obtenidas.
            \item Extraer con o sin remplazamiento bolas de una urna que contiene 5 bolas blancas y 6 negras.
            \item Contar el número de piezas defectuosas producidas por una máquina $X$ en un día determinado.
            \item Elegir un presidente de un grupo de 50 personas.
            \item Observar le tiempo de vida de una refrigeradora.
        \end{itemize}
        \subsubsection{Caracterización de un Experimento Aleatorio}



        








